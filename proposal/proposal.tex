%
% File naaclhlt2018.tex
%
%% Based on the style files for NAACL-HLT 2018, which were
%% Based on the style files for ACL-2015, with some improvements
%%  taken from the NAACL-2016 style
%% Based on the style files for ACL-2014, which were, in turn,
%% based on ACL-2013, ACL-2012, ACL-2011, ACL-2010, ACL-IJCNLP-2009,
%% EACL-2009, IJCNLP-2008...
%% Based on the style files for EACL 2006 by 
%%e.agirre@ehu.es or Sergi.Balari@uab.es
%% and that of ACL 08 by Joakim Nivre and Noah Smith

\documentclass[11pt,a4paper]{article}
\usepackage[hyperref]{naaclhlt2018}
\usepackage{times}
\usepackage{latexsym}
\usepackage{booktabs}
\usepackage{amsmath}
\usepackage{amsfonts}
\usepackage{amssymb}
\usepackage{microtype}
\usepackage{graphicx}

\usepackage{url}
\usepackage{color}

\aclfinalcopy % Uncomment this line for the final submission
%\def\aclpaperid{***} %  Enter the acl Paper ID here

%\setlength\titlebox{5cm}
% You can expand the titlebox if you need extra space
% to show all the authors. Please do not make the titlebox
% smaller than 5cm (the original size); we will check this
% in the camera-ready version and ask you to change it back.

\newcommand\BibTeX{B{\sc ib}\TeX}

\title{Personal Financial Data Mining Proposal \\ {\small Version 1.0}}

\author{Mark Kurzeja}

\date{}

\begin{document}
\maketitle
\begin{abstract}

For the last five-plus years, as a part of my monthly budgeting, I've been collecting information about every single dollar that I have ever spent during my times in junior year of college, senior year of college, my time living in Manhattan, and through my first and second year of graduate school. During this time, I found that budgeting and the prediction of even my own personal finances have been incredibly difficult and sometimes downright frustrating. While most of the time, the consequences for not budgeting exactly can be relatively minimal, there have been moments in my life where prediction errors have resulted in the times that have caused financial difficulties. Because of the emotional and financial difficulties that these shortfalls can cause, I have tried over the years to create simple metrics to predict upswings in spending and prevent myself from overspending. My goals for this project are twofold, 1) I would like to create better predictions of my own financial spending on variable items such as eating out, entertainment, and other expenses that are within my control, and 2), I would like to synthesize these into in index or prediction scale that I can use to monitor my current financial state which acts as a forward indicator of potential financial shortfalls in the future.
  
\end{abstract}

\section{Introduction}

\subsection{The Problem}

Managing one's own finances can be hard. It is no secret that the average American is rather terrible with money. It is estimated that as much as 20\% of the population has a negative net worth \cite{BNP}, and during my time working in the financial services industry, I've seen the results of many poor financial decisions completely wreck people with incomes  tens of times what the average American makes. I find myself to be fortunate that I have been able to maintain somewhat strict rules for myself, and for the past five or so years I have been tracking every single dollar that I've ever spent by hand. I've done this via the use of a budgeting app called YNAB. There have been times that I have been very good at estimating my future expenses, and when this happens I'm better able to predict my financial picture during the coming months. Especially whenever I have been in college, and in grad school, there have been times that I've had to go almost a full year without a paycheck. Whenever I lived in Manhattan, large expenses such as rent meant that if I miscalculated my other expenditures, I would not have enough money to do simple things such as make it back home for the holidays or see my girlfriend at the time. 

\subsection{First and Second Order Goals}

Sometimes however the consequences of not predicting your financial picture accurately can be devastating. we live day today not really knowing what tomorrow will hold, but whenever it comes to money, ignorance is anything but bliss. Errors in estimation can result in severe financial difficulties for some, and my hope with this project is to be able to provide a far more accurate estimate of my finances in the future using information about my spending patterns today and my spending patterns in the past. My hope is that my research into my own finances will be generalizable into that of my family which actually tracks all their finances in the exact same manner that I do. I hope that I can expand the system and one day I actually hope to own my own Financial advisory firm. My research into this topic, and the methods that I devised, I hope will one day be useful in predicting financial inflows and outflows in an effort to help people get a better track of their financial futures and ensure that they are able to achieve their goals despite the fact that the world is inherently chaotic, messy, and sometimes downright unpredictable.

\section{Problem Definition and Data}
\label{sec:problem}

My goal for this project is twofold: 
\begin{enumerate}
	\item I want to create a system of models that are able to better predict my reoccurring variable spending habits using his little of data as possible to get the most accurate predictions on a monthly basis.  I want to use as little of data as possible in order to ensure that the system can work for other people who do not have a large financial history such as I do. And I want the system to be is accurate as it possibly can given the information that I put in.
	\item I want to distill this prediction mechanism into an index they can track how people are doing overtime. Like a performance gauge, I want someone to know when they are doing well and are spending under what we would expect them to be spending at any given moment, and I want them to know whenever they are spending poorly and they need to curtail their current expenditures in order to get back on track. I think the simple metrics such as financial ratios aim to do this four people in a hand calculation sort of way. My aim is to make something that someone can track “with their eyes” so that they can see if they are doing well and so that they can manage their finances eventually on their own without the need of a financial adviser to tell them if they are doing well or not. 
\end{enumerate}

This project is not solely focused on predictions for one month out. This project is also not simply about forecasting. The models that I wish to build I want to be able to derive insights from. I want to be able to take the results and put them in the words and explain to people why the model may have over or under predicted in a given month, and why it thinks that its predictions are going to be on track. 

My evaluation metrics will include asymmetric L1 loss (the losses resulting from overprediction and a shortfall of money are far less severe than the losses resulting from foregone investment interest resulting from under budgeting and capital surplus). I want to run a conjoint analysis to determine what loss I can create that best aligns with how I experience loss. For this, I will place a large number of scenarios in front of myself and I want to evaluate which I would prefer in a given circumstance. For example, would I prefer to over budget by \$100 or would I prefer to under-budget vacation by \$80 and not be able to go out to dinner on the last night? Answers to questions such as these will be very useful in determining how I should assign loss measures and to which categories of my budget I should be the most focused on. 

\subsection{The Data Set}
I have every single dollar tallied that I have spent in the past five years in my budgeting program. The data for a transaction includes the following:
\begin{description}
	\item[Date] The Date the transaction took place
	\item[Account] From which of my personal accounts did the money come from? Checking, Savings, Etc?
	\item[Master Category] A high-level category assigned by myself. This includes things like is this a Housing Expense? Is this a Leisure Expense? Is this a Financial Expense? etc.
	\item[Sub Category] This is a more granular category that is meant to break up the master categories. If something is a housing expense, is it for rent, or household items, or for a security deposit?
	\item[Payee] Who was the transaction to?
	\item[Outflow] If the item was an outflow, what was the amount?
	\item[Inflow] If the item was an inflow, what was the amount?
	\item[Memo] User-generated description of the expense
\end{description}

Further to each of the datapoints above, I have the total amount spent in each category during the last five years and I can see how I planned for expenses in the future using the data from my budgeting values. 

Some of the challenges that I see from working with this data include:
\begin{description}
	\item[Sparsity] The fact that I only record a transaction once it happens and not when I expect it to occur means that for certain categories that are less frequent, there could be very little data predicting incidence in the future. For instance, I have only taken two vacations in the last three years which means that months were vacations are in the mix will be mostly zero expenditaure with some very large outliers
	\item[Zero-inflated models] To the sparsity point, most of the entries in the data are zero. Since some events only occur every other week or similar, I have to model both the non-occurrence of an event as well as the expectation when it does occur.
	\item[Privacy] This model will have to be anoymonized to a certain extent since this is a complete account of my financial picture. As a result, some censoring or name replacement may be necessary depending on the features that I find to be useful in predicting outflows.
\end{description}

%The section describes what specific problem you plan to solve and goes into much
%more detail than the Introduction.  You can also add details about what your
%problem is not to help guide the reader's expectations.  You should also go into detail
%about how you define success, i.e., what are the evaluation metrics you will use to evaluate.  How will you know if you have a good solution to
%your problem?

%Second, you should describe what data you will use for the project.  Ideally, you should already have the data on hand or spend a few minutes getting it.  Projects are \textit{much} more successful when most of the time is spent solving the problem rather than searching for the right data.  If you don't have data at proposal time, please be very specific on how you plan to get it.  We \textit{might} be able to recommend something but we encourage you to come to talk to us during office hours or after class.  If you have the data, you should include a few examples and can also include some very rough statistics (e.g., how many instances you have).

%Since not everyone has a burning question they're dying to answer with data mining, we've included a few  potential data sets you could work on in Table \ref{tab:tasks}.  For these datasets, you'll need to design a particular question you want to answer and how you answer it.  For students wanting off-the-shelf data mining tasks, Kaggle will have some challenges you can potentially use.  Their search engine is a good place to start.  

%\textbf{Important Note}: For the final project submission, student-proposed data mining tasks and existing tasks are evaluated slightly differently.  If you choose an existing data mining task for your project (e.g., from Kaggle), you will be expected to submit a final report that has a complete solution (a working algorithm) for the task.  This expectation does not mean you have to have a high-performing or highly novel system, just that you have completed the project.  If you pitch your own problem to work on, we realize that sometimes a task can be a bit more daunting than expected, or the data might not be what you expected.  In these cases, you're expected to report in detail how far you got and what went right and what went wrong.  Ideally, in these negative results papers, you should still be presenting a substantial analysis of the problem or data, it's just that this analysis doesn't ultimately lead to a solution.  Also, the instructors for this course will do whatever possible to help guide you in your projects so you don't get stuck!

%\begin{table*}
%\centering
%% NOTE: You normally don't need the resizebox for a table, but it's just so we can smush the table and URLs into one page.
%        \resizebox{\textwidth}{!}{  
%
%\begin{tabular}{lll}
%	\hline
%	Dataset                         & Description                        & Website                                                                    \\ \hline
%	MIMIC                           & 40K Deidentified Health Records    & {\small \url{https://mimic.physionet.org/} }                               \\
%	StudentLife                     & Longitudinal Student Data          & {\small \url{http://studentlife.cs.dartmouth.edu/dataset.html}}            \\
%	AAN                             & Citation Data and Metadata         & {\small \url{http://clair.eecs.umich.edu/aan/index.php}}                   \\
%	Wikipedia                       & Edits, comments, links, ...        & {\small{\url{https://en.wikipedia.org/wiki/Wikipedia:Database_download} }} \\
%	Reddit                          & Comments, posts, URLs, threads ... & {\small \url{https://www.reddit.com/r/datasets/comments/3bxlg7/} }         \\
%	\emph{More to posted on Piazza} &  \\ \hline
%\end{tabular}
%}
%\caption{Examples of datasets that you could choose to work on for your project.  }
%\label{tab:tasks}
%\end{table*}

%\textbf{\color{red} For the proposal, you should be have a clearly stated problem and a very high-level description of the data with a few examples shown.}

\begin{table*}
	\resizebox{\textwidth}{!}{  
	\begin{tabular}{cccccccc}
		        Account         &   Date    &     Payee     & Master Category &        Sub Category        &      Memo       & Outflow & Inflow \\ \midrule
		Capital One Quicksilver & 8/13/2016 & Diner Airport &  Everday Wants  & Partying, Drinks, and Food &  Food airport   & \$24.04 & \$0.00 \\
		Capital One Quicksilver & 8/13/2016 &   Quik Fill   &  Everday Wants  & Partying, Drinks, and Food &    Trip tea     & \$1.38  & \$0.00 \\
		    Cash In Wallet      & 8/20/2016 & Jakes Saloon  &  Everday Wants  & Partying, Drinks, and Food & Justin and Pete & \$27.00 & \$0.00 \\ \bottomrule
	\end{tabular}
	}
	\caption{Three Data points from the data set}
\end{table*}



\section{Related Work}

There has been a lot of work on personal forecasting but most of it, to maintain simplicity, has been based off of using simple averages and running averages to predict the spending in the coming months. 

More advanced methods of prediction have, to my knowledge, never been used on personal financial data in this fashion before, and so I am to bring the more advanced analytical toolbox to the personal financial space in an effort to improve the forecasting accuracy and budgeting prediction. While methodologies are probably implemented in the private wealth space for ultra-high net worth clients, the people that need this prediction setup the most are the people who don't have the money to afford the inaccuracies that budgeting can allow in the first place. 

The predicion problem for time series is well studied however, and from that perspective there are a variety of methods that have been brought to the table. Anything from ARIMA modeling to Facebook's Prophet have been found to be useful. Gaussian Processes are something that may be useful to get variance estimates for each of the predictions, and other models that allow for smoothing of noisy data may be useful as well. ASAP, which we saw in class, is something that may be able to take out a large variety of the variation that financial data often brings about as well. 




%The related work section should describe how other people have thought about the
%problem you're working on.  How did they approach it?  What makes their problem
%different from yours?  Why do you think your approach will be better?

%\textbf{\color{red} You should have \textit{at least three papers} related to your current problem and a few sentences describing what they did to solve the problem.  Since you haven't tried solving the problem yet, you don't need to compare with them at all.}



\section{Methodology}

Some of the methods that I plan to try for the forecasting piece of the assignment are as follows:
\begin{enumerate}
	\item As a baseline, how good / bad are running averages at predicting the future for financial data?
	\item ARIMA as a baseline
	\item Baysian Gaussian Processes as a means of modeling time series data with uncertainty estimates
	\item Baysian Hierarchial Modeling to predict the distributions of 1) the occurance of a day with a particular type of spending. i.e. did I buy groceries today? and 2) if spending occurs, what is the distribution of the spending? i.e. I'm going to the grocery store, what am I going to spend?
	\item Facebook's Prophet - I've heard great things about the sucess of this system in predicting time series. I'm curious how well it will perform on the sparse data that I have
	\item Smoothing techniques like filtering - sometimes noise reduction allows one to see the forest through the trees - I'm hopeful that something like ASAP or DFT will allow the noise that is present in a time series to be reduced for financial data enough to make robust predictions. 
	\item Sequence processing such as a hidden markov model may be very helpful as well and it is something that I would love to try
\end{enumerate}


Some of the methods that I plan to try for the metric piece of the assignment are as follows:
\begin{enumerate}
	\item Monthly Spending Index - at any given time, how much do I expect to spend in the next 30 days? This metric could be updated in real time to show an index of when I have been overspending or underspending
	\item Bullet charts to show good-caution-danger zone estimates for certain categories. Popularized by Stephen Few, these are great little charts and are highly dense information wise and can be very quickly made into a dashboard {\color{blue} \url{https://en.wikipedia.org/wiki/Bullet_graph}}
	\item Pain Index - Given the conjoint information from the beginning assessment, how likely are you to experience pain in the form of under-budgeting within the next month? Tracking this over time will let someone know how close they typically are to their goal and provide course correction in the event that they are unaware that they are not on track. 
\end{enumerate}


\section{Evaluation and Results}

From above: My evaluation metrics will include asymmetric L1 loss (the losses resulting from overprediction and a shortfall of money are far less severe than the losses resulting from foregone investment interest resulting from under budgeting and capital surplus). I want to run a conjoint analysis to determine what loss I can create that best aligns with how I experience loss. For this, I will place a large number of scenarios in front of myself and I want to evaluate which I would prefer in a given circumstance. For example, would I prefer to over budget by \$100 or would I prefer to under-budget vacation by \$80 and not be able to go out to dinner on the last night? Answers to questions such as these will be very useful in determining how I should assign loss measures and to which categories of my budget I should be the most focused on. 

In math, one such metric would be as follows (let $R$ be the realized amount of spending and $B$ be the amount Budgeted):
\begin{align}
Loss(R, B) &= w_1 \max(R - B, 0) \nonumber \\
&+ w_2 \max(B - R, 0)
\end{align}

Where, if $realized > budgeted$, then we have overspent and $w_1$ will drive our loss and the converse holds true for $w_2$. Given our discussion from before, it makes sense that $w_1 \geq w_2$ since the pain of having a shortfall, for most, will be greater than the gain that one experiences from optimizing their free cash flow. However, when I run the conjoint analysis, I hope to find out if this is really the case. 

The baseline metrics that I will use are both very common in the space:
\begin{enumerate}
	\item Running (or Simple) Average - Simple means and measures of center are the most common implementation for many of the predcition metrics from month to month. This is the method that is used in my budgeting program currently, and its inadequacy is the main reason why I thought of this project. 
	\item ARIMA - this is probably the most sophiisticaed model that most anyone who works with basic finaical data uses since it is widely implemented and this is usually the most advanced (and only) mention of time series in most business schools\footnote{During my time in Ross, this was only mentioned in an advanced analytics course at the 400 level and was a highly non-standard part of the curriculium. During my time in banking, most forecasting was done at a far more micro level using correlations and so large scale time series analysis were left to the research economists who used mean expectation algorithms for stabilty.}. 
\end{enumerate}




%This section provides an overview of how you evaluated your method on the data.  What methods did you compare against?  How successful were you?  Describe the exact evaluation setup and what kinds of
%steps were taken.

%You should also clearly define some baselines to compare your system against.
%One baseline should be random performance.  A second baseline should be
%something reasonable that doesn't require much knowledge or learning.  For example, if you're doing a recommendation engine, always choosing the average rating is a useful baseline.

%\textbf{ \color{red} For the proposal, you should propose a very simple baseline to compare your model against and a description of the evaluation metric(s) you intent to use.  You \textit{must} include this baseline description for credit, as it is critical to  proposing an evaluatable data mining task.}

%
%\begin{figure}
%\centering
%% Note: text width is the width of both columns
%%\includegraphics[width=0.38\textwidth]{examplefigure.pdf}
%\caption{Eventually, you'll have cool figures to show here.}
%\end{figure}

%\section{Discussion}



%The discussion section is where you start to unpack the results for the reader
%to help them understand what was learned.  You may not have many results at the
%moment (but you should have some!) so you can discuss here what has gone wrong
%and right in your current setup.  For example, maybe you realized you needed
%more data, or maybe you realized that the SVD was not helping your analysis.
%The discussion should point the way to what work will be done next.


%\textbf{ \color{red} You can leave this section blank. }




\section{Work Plan}

My plan for this project roughly follows the following guideline:
\begin{enumerate}
	\item Gather the data from my budgeting program. I already have a workflow for this
	\item Determine which categories are worth modeling - I'm looking at any variable expenses as being the most useful thing to model. Fixed expenses like rent are contractual and can be modeled statically
	\item Begin building look-ahead prediction models and comparing them. 
	\begin{enumerate}
		\item Start with base models. Assess prediction accuracy
		\item Work up to more advanced models
		\item Try to combine results using stacking to improve prediction error
		\item Reduce data-size as much as possible to see how much data we need for ``good enough'' accuracy
	\end{enumerate}
	\item Begin looking at metrics that assess the health of a persons spending as a means of visualizing progress
	\begin{enumerate}
		\item Track financial ratios and see if they are ``useful''
		\item Look at smoothing procedures to see if they add information
		\item Look at simulatoin based visualizations to see if they add anything interesting
	\end{enumerate}
\end{enumerate}

%\textbf{ \color{red}  Normally, you would have a conclusion to end the work, but as this is a project proposal,
%you'll end with a plan for how you'll accomplish your project.  We hope this section can help you think at a high-level about which tasks are necessary to get to the point where you have a working model.  Of course, plans are often made and then changed when new information or challenges emerge, so we won't hold you you to this.  However, the act of writing a plan can greatly help you figure out how think about the process and, in general, projects that are proposed with more concrete work plans tend to be more successful.
%}

%\section{Multi-person Team Justification}
%
%\textbf{ \color{red} If you have more than one person on your project, you should justify why the work requires the number of people you have.  In addition, you should provide a concrete explanation of what each team member is expected to do, keeping in mind that all team members need to be doing \textit{some} kind of data mining for the project.}

\section*{Acknowledgments}

My brother, an accountant, and I as well as several friends have discussed beginning a financial advisory firm focusing on tax arbitrage and investing advice based on tailored risk analysis. My background in finance and financial modeling heavily informs my views of financial predictions in general, and I know how noisy the real world is with this kind of data. Having always been interested in personal finance, this aspect of modeling out my finances is something I have attempted in the past using various simulation based methods, but I have never taken a deep dive into quantifying error nor coming up with a system for prediction and analysis. I hope to expand this for the ambitions of a past self and for the benefit of a potential business

%If you got help from anyone or had substantive discussions, please acknowledge
%those people here and describe how they contributed.  The work you do for your
%project should be entirely your own.

% include your own bib file like this:
%\bibliography{references}
%\bibliographystyle{acl_natbib}

%\textbf{\color{red} Note that you must cite all your references, e.g., \cite{wang2014studentlife}}

%\appendix

\begin{thebibliography}{9}
	\bibitem{BNP} 
	BNP Paribas - During my time at BNP Paribas in New York, this was often a statistic that was used in client meetings and for presentations. 
	
	
	
\end{thebibliography}


%\section{Supplemental Material}
%\label{sec:supplemental}
%
%If you want to put longer examples of data and code, put it here in the appendix.  

\end{document}
