%
% File naaclhlt2018.tex
%
%% Based on the style files for NAACL-HLT 2018, which were
%% Based on the style files for ACL-2015, with some improvements
%%  taken from the NAACL-2016 style
%% Based on the style files for ACL-2014, which were, in turn,
%% based on ACL-2013, ACL-2012, ACL-2011, ACL-2010, ACL-IJCNLP-2009,
%% EACL-2009, IJCNLP-2008...
%% Based on the style files for EACL 2006 by 
%%e.agirre@ehu.es or Sergi.Balari@uab.es
%% and that of ACL 08 by Joakim Nivre and Noah Smith

\documentclass[11pt,a4paper]{article}
\usepackage[hyperref]{naaclhlt2018}
\usepackage{times}
\usepackage{latexsym}
\usepackage{graphicx}

\usepackage{url}
\usepackage{color}

\aclfinalcopy % Uncomment this line for the final submission
%\def\aclpaperid{***} %  Enter the acl Paper ID here

%\setlength\titlebox{5cm}
% You can expand the titlebox if you need extra space
% to show all the authors. Please do not make the titlebox
% smaller than 5cm (the original size); we will check this
% in the camera-ready version and ask you to change it back.

\newcommand\BibTeX{B{\sc ib}\TeX}

\title{Instructions for the SI 671/721 Project Proposal \\ {\small Version 1.0}}

\author{Mark Kurzeja}

\date{}

\begin{document}
\maketitle
\begin{abstract}

This document summarizes the main parts of the project proposal, which sections are needed, and what's needed in them.  Your task will be to fill in these parts, which are marked in red.  There is additional commentary to help you in the scientific writing process.  The proposal, progress update, and final submission all build upon one another so the work you put into the proposal will directly carry over to each and help provide a roadmap of what's to be done.

\textbf{\color{red} For now, your abstract should just summarize what's the problem you're working on.}
  
  
\end{abstract}

\section{Introduction}

The course project is intended to provide an opportunity for students to dive deeper into one problem or topic of their choice and write a very small scale study on the topic.  Projects typically take two forms: (1) the student has some data, problem, or algorithm in mind and proposes a study to investigate these or (2) students pick an existing data mining task and try a new approach to solving it.  Tasks for the latter are detailed more in Section \ref{sec:problem}.  In both cases, projects should be \textit{feasible} for completing in the available time frame.  The latter part of the course has a lighter workload to allow more time for working on the project, but we want to ensure that students pick projects that help them learn real, practical skills in data mining without being trivial.  Ideally, your course project is a chance to develop something you can show off to future employers or could serve as a pilot study for a full research project.

For this project, you're expected to use this \LaTeX template.  You're welcome to copy this template directly off of Overleaf as well using this URL: \url{https://www.overleaf.com/read/jprvhgvckpxz}, which hopefully has enough examples of how things can be written to get you started.  If you're having any issues getting \LaTeX to do what you want please feel free to ask the instructors or know that there's a great resource on WikiBooks \url{https://en.wikibooks.org/wiki/LaTeX} and a whole StackExchange site dedicated to answering questions \url{https://tex.stackexchange.com/}.   \LaTeX is a common method for writing technical documents so we are using it for 671 to help get you started on using it for your career.  It also is pretty awesome for citation management.

Finally, please remember that the 671 instructors are here for you and will gladly offer suggestions and advice on projects.  We want your projects to succeed, to be fun to work on, and to spark your intellectual curiosity!  


\textbf{\color{red} You should have a rough draft of the introduction that clearly states what the problem is and provides some broader context.  We recommend writing the introduction last after you finish the Problem Definition and Related Works sections.  You should also include a statement on why solving this problem matters--who would care if you solved it and what effect would solving it have?}

The SI671/721 projects will all be presented at the winter symposium, which is a poster session showing off the great work done by UMSI students.  We expect to have many community representatives, potential employers and recruiters, and other faculty and students present.  As an eye to the future, in the proposal, think about who you would want to see your work and why your results would matter to them.  \textbf{\color{red} Specifically, in the introduction, write a short paragraph outlining which persons or organizations would be interested to see your project and why any potential outcomes might be exciting to them.  Be realistic in your potential outcomes but assume that your project will go according to plans. }

\section{Problem Definition and Data}
\label{sec:problem}

The section describes what specific problem you plan to solve and goes into much
more detail than the Introduction.  You can also add details about what your
problem is not to help guide the reader's expectations.  You should also go into detail
about how you define success, i.e., what are the evaluation metrics you will use to evaluate.  How will you know if you have a good solution to
your problem?

Second, you should describe what data you will use for the project.  Ideally, you should already have the data on hand or spend a few minutes getting it.  Projects are \textit{much} more successful when most of the time is spent solving the problem rather than searching for the right data.  If you don't have data at proposal time, please be very specific on how you plan to get it.  We \textit{might} be able to recommend something but we encourage you to come to talk to us during office hours or after class.  If you have the data, you should include a few examples and can also include some very rough statistics (e.g., how many instances you have).

Since not everyone has a burning question they're dying to answer with data mining, we've included a few  potential data sets you could work on in Table \ref{tab:tasks}.  For these datasets, you'll need to design a particular question you want to answer and how you answer it.  For students wanting off-the-shelf data mining tasks, Kaggle will have some challenges you can potentially use.  Their search engine is a good place to start.  

\textbf{Important Note}: For the final project submission, student-proposed data mining tasks and existing tasks are evaluated slightly differently.  If you choose an existing data mining task for your project (e.g., from Kaggle), you will be expected to submit a final report that has a complete solution (a working algorithm) for the task.  This expectation does not mean you have to have a high-performing or highly novel system, just that you have completed the project.  If you pitch your own problem to work on, we realize that sometimes a task can be a bit more daunting than expected, or the data might not be what you expected.  In these cases, you're expected to report in detail how far you got and what went right and what went wrong.  Ideally, in these negative results papers, you should still be presenting a substantial analysis of the problem or data, it's just that this analysis doesn't ultimately lead to a solution.  Also, the instructors for this course will do whatever possible to help guide you in your projects so you don't get stuck!

\begin{table*}
\centering
% NOTE: You normally don't need the resizebox for a table, but it's just so we can smush the table and URLs into one page.
        \resizebox{\textwidth}{!}{  

\begin{tabular}{lll}
	\hline
	Dataset                         & Description                        & Website                                                                    \\ \hline
	MIMIC                           & 40K Deidentified Health Records    & {\small \url{https://mimic.physionet.org/} }                               \\
	StudentLife                     & Longitudinal Student Data          & {\small \url{http://studentlife.cs.dartmouth.edu/dataset.html}}            \\
	AAN                             & Citation Data and Metadata         & {\small \url{http://clair.eecs.umich.edu/aan/index.php}}                   \\
	Wikipedia                       & Edits, comments, links, ...        & {\small{\url{https://en.wikipedia.org/wiki/Wikipedia:Database_download} }} \\
	Reddit                          & Comments, posts, URLs, threads ... & {\small \url{https://www.reddit.com/r/datasets/comments/3bxlg7/} }         \\
	\emph{More to posted on Piazza} &  \\ \hline
\end{tabular}
}
\caption{Examples of datasets that you could choose to work on for your project.  }
\label{tab:tasks}
\end{table*}

\textbf{\color{red} For the proposal, you should be have a clearly stated problem and a very high-level description of the data with a few examples shown.}



\section{Related Work}

The related work section should describe how other people have thought about the
problem you're working on.  How did they approach it?  What makes their problem
different from yours?  Why do you think your approach will be better?

\textbf{\color{red} You should have \textit{at least three papers} related to your current problem and a few sentences describing what they did to solve the problem.  Since you haven't tried solving the problem yet, you don't need to compare with them at all.}



\section{Methodology}

This section will describe how you solve your problem and mine your data.  Go into algorithmic details and be sure to describe what various kinds of preprocessing steps you
plan to do.  In the final report, someone should be able to recreate your exact methodology from the
description.  Be specific about what each step does.  For example, it's
insufficient to say ``we ran and SVD;'' instead say something like ``we
constructed a matrix where songs are rows and plants are columns and growth rates are indicated in each cell; then we ran an SVD to find regularities in song-plant growth.''

As a part of this section, please describe \emph{why} you chose what you did.
What was your design process and how did you approach solving the problem?

\textbf{\color{red} 
For the proposal, you should include a  very general description of what method(s) you plan to try.  }


\section{Evaluation and Results}

This section provides an overview of how you evaluated your method on the data.  What methods did you compare against?  How successful were you?  Describe the exact evaluation setup and what kinds of
steps were taken.

You should also clearly define some baselines to compare your system against.
One baseline should be random performance.  A second baseline should be
something reasonable that doesn't require much knowledge or learning.  For example, if you're doing a recommendation engine, always choosing the average rating is a useful baseline.

\textbf{ \color{red} For the proposal, you should propose a very simple baseline to compare your model against and a description of the evaluation metric(s) you intent to use.  You \textit{must} include this baseline description for credit, as it is critical to  proposing an evaluatable data mining task.}


\begin{figure}
\centering
% Note: text width is the width of both columns
%\includegraphics[width=0.38\textwidth]{examplefigure.pdf}
\caption{Eventually, you'll have cool figures to show here.}
\end{figure}

\section{Discussion}

The discussion section is where you start to unpack the results for the reader
to help them understand what was learned.  You may not have many results at the
moment (but you should have some!) so you can discuss here what has gone wrong
and right in your current setup.  For example, maybe you realized you needed
more data, or maybe you realized that the SVD was not helping your analysis.
The discussion should point the way to what work will be done next.


\textbf{ \color{red} You can leave this section blank. }




\section{Work Plan}

\textbf{ \color{red}  Normally, you would have a conclusion to end the work, but as this is a project proposal,
you'll end with a plan for how you'll accomplish your project.  We hope this section can help you think at a high-level about which tasks are necessary to get to the point where you have a working model.  Of course, plans are often made and then changed when new information or challenges emerge, so we won't hold you you to this.  However, the act of writing a plan can greatly help you figure out how think about the process and, in general, projects that are proposed with more concrete work plans tend to be more successful.
}

\section{Multi-person Team Justification}

\textbf{ \color{red} If you have more than one person on your project, you should justify why the work requires the number of people you have.  In addition, you should provide a concrete explanation of what each team member is expected to do, keeping in mind that all team members need to be doing \textit{some} kind of data mining for the project.}

\section*{Acknowledgments}

If you got help from anyone or had substantive discussions, please acknowledge
those people here and describe how they contributed.  The work you do for your
project should be entirely your own.

% include your own bib file like this:
\bibliography{references}
\bibliographystyle{acl_natbib}

\textbf{\color{red} Note that you must cite all your references, e.g., \cite{wang2014studentlife}}

\appendix

\section{Supplemental Material}
\label{sec:supplemental}

If you want to put longer examples of data and code, put it here in the appendix.  

\end{document}
